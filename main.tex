\documentclass{article}

\usepackage{amsmath, amsthm, amssymb}

\newtheorem{thm}{Theorem}[section]
\newtheorem{cor}[thm]{Corollary}
\newtheorem{lem}[thm]{Lemma}

\title{..}
\author{Edward O'Callaghan}

\begin{document}
\maketitle

\section{Hilbert Space $L^2$}

In the special and unique case of fixed $p=2$ in $L^2(\mathbb{R})$ we obtain a Hilbert
space. Hilbert space $L^2(a,b)$ has the fruitful property that polynomials over
it form a dense subspace.

Let us define the Banach space $L^p(\mathbb{R})\, \forall p \in [1, \infty[$ and then
restrict ourselves to the case of $p=2$ to define the space:
\[
 L^2(\mathbb{R}) = \left\{ f : \mathbb{R} \rightarrow \mathbb{C} : \int_{-\infty}^{\infty} \left| f(x) \right|^2 dx < \infty \right\}.
\]
Herein we show that this space can indeed be equipped with an inner product.

\begin{thm}
The vector space $L^2(\mathbb{R})$ is a Hilbert space with respect to the inner product:
  \[
    \langle f,g \rangle = \int_{-\infty}^{\infty} f(x) \overline{g(x)} \, dx, \, f,g \in L^2(\mathbb{R}).
  \]
\end{thm}

\begin{proof}
We know that, for each $p\in[1,\infty[$, the expression
  \[
    \|f\|_p = \left( \int_{-\infty}^{\infty} |f(x)|^p \, dx \right)^{1/p}
  \]
defines a norm on $L^p(\mathbb{R})$ that makes $L^p(\mathbb{R})$ a Banach
space, and so the special case of $p=2$ it trivially follows $L^2(\mathbb{R})$
is Banach.

The integral in the inner product is well defined when the map $x \mapsto f(x)
\overline{g(x)}$. is within $L^1(\mathbb{R})$ and hence we are required to prove
this. By using Holder's inequality, fixing $p=q=2$, we observe $\forall f,g \in
L^2(\mathbb{R})$ that,
\[
  \int_{-\infty}^{\infty} |f(x) \overline{g(x)}| \, dx \leq
  \left( \int_{-\infty}^{\infty} |f(x)|^2 \, dx \right)^{1/2}
  \left( \int_{-\infty}^{\infty} |g(x)|^2 \, dx \right)^{1/2} < \infty
\]
and so the required results immediately follow.
\end{proof}

\end{document}
